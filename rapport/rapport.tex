\documentclass[a4paper, 12pt, french]{article}
\usepackage{ae,lmodern}
\usepackage[french]{babel}
\usepackage[utf8]{inputenc}
\usepackage[T1]{fontenc}
\usepackage[dvipsnames]{xcolor}
\usepackage{graphicx}
\usepackage{hyphenat}
\usepackage[left=15mm, right=15mm]{geometry}
\usepackage{csquotes}
\usepackage{bookmark}
\usepackage{biblatex}
\usepackage{listings}
\usepackage{hyperref}
\usepackage{eurosym}

\addbibresource{rapport.bib}

\lstset{
  aboveskip=3mm,
  belowskip=-2mm,
  backgroundcolor=\color{lightgray},
  basicstyle=\footnotesize,
  breakatwhitespace=false,
  breaklines=true,
  captionpos=b,
  commentstyle=\color{ForestGreen},
  deletekeywords={\ldots},
  escapeinside={\%*}{*)},
  extendedchars=true,
  framexleftmargin=16pt,
  framextopmargin=3pt,
  framexbottommargin=6pt,
  frame=tb,
  keepspaces=true,
  keywordstyle=\color{blue},
  language=Python,
  literate=
  {²}{{\textsuperscript{2}}}1
  {⁴}{{\textsuperscript{4}}}1
  {⁶}{{\textsuperscript{6}}}1
  {⁸}{{\textsuperscript{8}}}1
  {€}{{\euro{}}}1
  {é}{{\'{e}}}1
  {è}{{\`{e}}}1
  {ê}{{\^{e}}}1
  {ë}{{\¨{e}}}1
  {É}{{\'{E}}}1
  {Ê}{{\^{E}}}1
  {û}{{\^{u}}}1
  {ù}{{\`{u}}}1
  {â}{{\^{a}}}1
  {à}{{\`{a}}}1
  {á}{{\'{a}}}1
  {ã}{{\~{a}}}1
  {Á}{{\'{A}}}1
  {Â}{{\^{A}}}1
  {Ã}{{\~{A}}}1
  {ç}{{\c{c}}}1
  {Ç}{{\c{C}}}1
  {õ}{{\~{o}}}1
  {ó}{{\'{o}}}1
  {ô}{{\^{o}}}1
  {Õ}{{\~{O}}}1
  {Ó}{{\'{O}}}1
  {Ô}{{\^{O}}}1
  {î}{{\^{i}}}1
  {Î}{{\^{I}}}1
  {í}{{\'{i}}}1
  {Í}{{\~{Í}}}1,
  morekeywords={*,self, \_\_init\_\_, \_\_eq\_\_, \_\_str\_\_},
  numbers=left,
  numbersep=10pt,
  numberstyle=\tiny\color{black},
  rulecolor=\color{black},
  showspaces=false,
  showstringspaces=false,
  showtabs=false,
  stepnumber=1,
  stringstyle=\color{ForestGreen},
  tabsize=4,
  title=\lstname,
}

\title{
	\Huge
	\textbf{Développement d'une méthode de recherche arborescence pour un jeu à deux joueurs:
		application au jeu 7 Wonder-Duel}
	\vspace{0.4cm}

	\LARGE
	TER
}

\author{
	Cambresy Florian \\
	Chalaud Jean-Christophe \\
	Le Denmat Mickael
}

\begin{document}
	\begin{titlepage}
    \begin{center}
        \vspace*{1cm}

        \Huge
        \textbf{Développement d'une méthode de recherche arborescence pour un jeu à deux joueurs:
            application au jeu 7 Wonder-Duel}

        \vspace{0.4cm}
        \LARGE
        TER

        \vspace{1.6cm}

        \large
        CAMBRESY Florian \\
        CHALAUD Jean-Christophe \\
        LE DENMAT Mickaël \\

        \vfill

        % TODO : image projet à mettre ?
        % \includegraphics[width=0.27\textwidth]{image/logo.jpeg} \\
        \vspace{0.8cm}
        \includegraphics[width=0.32\textwidth]{images/UVSQ-logo}

        \vspace{0.4cm}

        \Large
        Université de Versailles Saint-Quentin-en-Yvelines \\
        \vspace{0.4cm}
        DATE ???
    \end{center}
\end{titlepage}

	\newpage
	\renewcommand{\contentsname}{Table des matières}
	\tableofcontents

	\newpage
	\section{Développement d'une méthode de recherche arborescence pour un jeu à deux joueurs:
	application au jeu 7 Wonder-Duel}
	\subsection{Sujet}
	Il s’agira tout d’abord pour les étudiants de se familiariser avec les méthodes de recherche arborescente
	appliquées aux jeux, en particulier la méthode alpha-bêta.

	Dans un second temps, il conviendra d'implémenter une telle méthode pour le jeu intitulé 7wonders - Duel.
	Ce jeu est décrit sur de nombreux sites, certains proposent même une petite vidéo tutorielle expliquant
	les règles de ce jeu. Il est à noter que l’encadrant offrira ce jeu aux étudiants qui
	choisiront ce sujet.

	Pour que le programme “joue bien”, il conviendra en particulier de réfléchir à des fonctions d'évaluation
	pertinentes des positions de jeu.

	\subsection{Description du travail attendu}
	L'objectif du projet est d'implémenter une fonction d'évaluation ainsi que la méthode alpha-bêta pour
	faire en sorte que qu'un algorithme puisse jouer. Ce dernier va donc prendre en entrée une situation de départ,
	les cartes de chaque joueur, leurs monnaies, ect et va fournir un coup à faire. De plus ce coups doit
	être pertinent et digne d'intérêt. Nous discuterons de cette définition par la suite.

	Afin d'apporter une solution à ce sujet, nous découperons le projet en trois parties.

	La première sera une partie descriptive concernant les notions contenues dans la méthode alpha-bêta pour
	le programme joue correctement en insistant sur les points qui devront être étudiés afin d'appliquer la méthode
	à notre exemple précis. Ensuite, nous expliquerons les règles du jeu et évoquerons le déroulement d'une partie.
	Nous développerons le système de victoire et, afin de faire une transition sur la partie qui suivra, nous
	débattrons concernant la force des "positions" au cours d'une partie.

	La deuxième partie montrera le chemin de pensée que nous avons eu afin d'arriver à une ou des solutions pour
	appliquer la méthode alpha-beta sur notre jeu. Plus particulièrement nous décrirons la fonction d'évaluation
	que nous avons trouvé afin que l'algorithme suggère des coups pertinents.

	Enfin la dernière sera une présentation des solutions techniques que nous avons mis en place.

	\subsection{Description du jeu : 7 wonders-Duel}
	7 Wonder est un jeu de plateau sorti dans les années 2010, créé par Antoine Bauza et publié par Repos Production en
	Belgique. Le nombre de joueurs est entre 3 et 7. On y joue une ancienne
	civilisation avec ses conflits militaires mais aussi ses activites commerciales. Il est connu et très apprécié
	par la communauté ayant remporté plus de 30 prix et souvent cité comme l'un des jeux de société les plus influents
	de la dernière décennie\cite{wiki_7_wonder}.
	Le jeu choisi afin d'appliquer l'algorithme min-max est une variante de celui-ci, le 7 wonders - Duel.

	\subsubsection{7 wonders - Duel}
	Le jeu 7 wonders - Duel est comme son nom l'indique un 7 wonders mais à deux joueurs. Un jeu sorti en 2015
	par la même société.

	\subsubsection{Règles du jeu}
	Le jeu se présente comme suit dans le livret des règles\cite{regle_7_wonder_duel}.

	 \includegraphics[width=15cm]{images/PlateauDebut}

	Il est constitué de:
	\begin{itemize}
		\item 1 plateau de jeu.
		\item 66 cartes Âge.
		\begin{itemize}
			\item 23 cartes pour l'Âge I.
			\item 23 cartes pour l'Âge II.
			\item 20 cartes pour l'Âge III.
		\end{itemize}
		\item 7 cartes Guilde.
		\item 12 cartes Merveille, avec un nom, un coût en ressource et un effet (un bonus).
		\item 4 jetons Militaire, donnant une certaine somme de monnaie.
		\item 10 jetons Progrès.
		\item 1 Pion Conflit, indiquant quel joueur a l'avantage militaire.
		\item 31 pièces de monnaie
		\begin{itemize}
			\item 14 de valeur 1.
			\item 10 de valeur 3.
			\item 7 de valeur 6.
		\end{itemize}
		\item 1 carnet de scores.
	\end{itemize}

	Les cartes Âge et Guilde sont des bâtiments. Elles peuvent avoir un coût (monetaire) afin d'être
	utilisées, qui est placé en haut à gauche. Elles peuvent donner des effets, placé en haut au centre.
	Les effets sont multiples comme par exemple une production de matière, une réduction de coût de construction
	d'un bâtiment ou d'une merveille, ainsi que d'autres. Enfin elles ont aussi un nom, placé en bas. Les cartes
	ont des couleurs différentes indiquant à quel type de bâtiment elles appartiennent.
	\begin{itemize}
		\item Les Cartes marrons sont des bâtiments de matière première.
		\item Les Cartes grises sont des bâtiments de produit manufacturé.
		\item Les cartes bleues sont des bâtiments civils, elles donnent des points de victoire.
		\item Les cartes vertes sont des bâtiments scientifiques, elles donnent aussi des points de victoire et
		\item Un symbole scientifique. Lorsqu'un joueur a deux symboles scientifique identique
		il peut prendre un jeton Progres.
		\item Les cartes oranges sont des bâtiments commerciaux, elles ont des objectifs multiples comme donner
		des pièces au joueur, produire des ressources, modifier les règles de commerce.
		\item Les cartes rouges sont des bâtiments militaires, elles augmentent la puissance.
		\item Les cartes violettes sont des bâtiment de Guilde, elles donnent des points de victoire en fonction
		de certains critères.
	\end{itemize}

	De plus le joueur peut acheter des ressources à la "banque" via le commerce comme par exemple si le joueur souhaite
	construire un bâtiment mais ne dispose pas de toutes les ressouces. Pour acheter cette ressource il faut prendre
	en compte les ressources produites par le joueur adverse (carte marron et grise) y ajouter 2 et remettre la somme
	dans la banque puis construire le bâtiment. Concernant les bâtiments, certains octroient un symbole de chaînage
	(en blanc en haut de la carte). Si le joueur souhaite construire un bâtiment et si il possède une carte
	donnant le symbole de chaînage il peut alors le construire gratuitement, dans le cas contraire
	il doit payer son coût en ressources et/ou monetaire.

	Une fois la présentation faite, nous allons pouvoir parler de la préparation du jeu avant de débuter une partie.
	La préparation du plateau se fait ainsi:
	\begin{enumerate}
		\item Placez le plateau entre les deux joueurs.
		\item Placez le pion Conflit sur la case neutre au milieu du plateau.
		\item Placez les quatres jetons Militaire sur leurs emplacements.
		\item Mélangez les jetons Progrès et placez-en cinqs au hasard sur la plateau.
		\item Chaque joueur prend 7 pièces à la banque.
	\end{enumerate}

	Puis vient la selection des Merveilles, pour cela il faut désigner le premier joueur et mélanger les Merveilles,
	ensuite:
	\begin{itemize}
		\item Disposez 4 Merveilles aléatoires entre les joueurs.
		\item Le premier joueur choisit une Merveille.
		\item Le deuxième joueur en choisit deux.
		\item Le premier joueur prend la dernière.
		\item Disposez les quatres autres Merveilles.
		\item Le deuxième joueur choisit une Merveille.
		\item Le premier joueur en choisit deux.
		\item Le deuxième joueur prend la dernière.
	\end{itemize}
	Pour construire une Merveille, le joueur doit mettre une carte face cachée sous la Merveille (peu importe la
	carte, elle sert juste a indiquer que la Merveille est construite) et doit payer le coût de la construction.
	Il y a un maximum de sept Merveilles construit par partie, le joueur dont la dernière Merveille n'est pas
	construite doit la remettre dans la boite.

	Enfin il faut placer les cartes d'Âge dans une structure particulière en fonction de l'âge. Les cartes face
	cachées sont retournées lorsque les cartes visibles posées dessus sont enlevées par les joueurs. Les joueurs
	peuvent choisir aussi de défausser une carte en échange de 2 pièces et d'un bonus d'une pièce pour chaque carte
	jaune qu'il possède.

	Il y a trois manières de gagner, la première est la victoire militaire, l'un des joueur aura avancé le pion
	Conflit dans la case capitale de son adversaire. La deuxième étant la victoire par suprémactie scientifique,
	l'un des joueurs réunit 6 symboles scientifiques différents. Enfin, si aucun des joueurs n'a gagné à la fin de
	l'Âge III, il faut alors faire la sommes des points de victoire acquis durant la partie,
	celui qui en possde le plus l'emporte.

	\section{Rappels généraux}
	\subsection{Théorie des jeux}
	Dans le monde des mathématiques il y a un domaine qui s'intéresse aux interactions stratégiques qui existent
	entre plusieurs agents. Ces interactions peuvent se faire dans un jeu, dans les sciences sociales, politiques
	ainsi que dans d'autres exemples. Ce domaine est la théorie des jeux\cite{wiki_theorie_jeux}.
	Il faut noter que ici le mot "jeu" a un sens plus large qu'un jeu de société ou autre.

	Dans ce projet nous porterons attention uniquement aux interactions entre deux joueurs dans un jeu de société à
	travers un exemple que nous développerons par la suite.

	\subsection{Catégories de jeux}
	Afin de connaître quelle approche doit être utilisée pour étudier les stratégies qui existent au sein d'un jeu,
	la théorie repartie les jeux en différentes catégories\cite{wiki_typo_theorie_jeux}.
	\begin{enumerate}
		\item Coopératifs ou non.
		\item Somme nulle ou non.
		\item Simultanés ou séquentiels.
		\item Information complète ou non.
		\item Mémoire parfaite ou non.
		\item Determiné.
		\item Finis.
		\item Répétés.
	\end{enumerate}
	(1) Un jeu coopératif permet d'analyser la formation de coalitions afin d'améliorer le gain potentiel. \\
	(2) Un jeu à somme nulle (ou jeu strictement compétitif) est un jeu dont les gains de certains joueurs
	sont les pertes des autres afin d'avoir la somme des gains et des pertes nulle, comme dans les échecs,
	dans le poker ou dans d'autres. A l'inverse dans un jeu à somme non nulle, tous les joueurs peuvent perdre,
	ou gagner, voir même certains peuvent gagner un gain moins (réciproquement plus) important que la perte totale
	d'autres joueurs.\\
	(3) Un jeu est dit "simultané" si les joueurs jouent en même temps, dans le cas contraire c'est un jeu séquentiel.\\
	(4) Un jeu peut être à "information complète", c'est-à-dire que chaque joueur connait l'ensemble des informations
	qui influent leur prise de décision.\\
	(5) Un jeu peut être aussi à "mémoire parfaite", dans ce cas chaque joueur peut se rappeler des coups qui ont été
	joués soit en les notant a part, soit ils sont visibles au cours de la partie (par exemple les cartes qu'a choisie
	le joueur sont à coté de ce dernier, face visible).\\
	(6) Un jeu "déterminé" est un jeu particulier à somme nulle sans intervention du hasard.

	\subsection{Représentation d'un jeu}
	Maintenant que nous savons comment définir un jeu, nous allons étudier comment représenter ce dernier et les
	interactions entre les joueurs. Il existe globalement trois représentations\cite{wiki_representation_theorie_jeux},
	la forme extensive et la forme normale, qui sont utilisées pour les jeux non coopératifs et
	la forme caractéristique pour les jeux coopératifs.

	\subsubsection{forme extensive}
	La forme extensive est la représentation sous forme d'arbre de décision où les noeuds d'une même hauteur sont les
	choix possibles (qui ne sont pas en dehors des règles prévues par le jeu) d'un joueur. Plus précisément chaque
	noeud est une situation produite au cours d'une partie (pas nécessairement le début de la partie) et les fils de
	ce noeud sont les coups que peut faire le joueur suivant. Par exemple dans les échecs, la racine sera l'état du
	plateau de jeu à la fin du tour d'un joueur (la position des pions de chaque joueur) et les fils de cette racine
	seront les déplacements des pions de l'adversaire. Dans cet exemple, ainsi que dans tous les jeux à deux joueurs,
	les hauteurs paires sont les coups du joueur A tandis que les hauteurs impaires sont les coups du joueur B.
	De plus, les feuilles de cet arbre sont des noeuds indiquant la fin de la partie, soit il y a victoire d'un des
	deux joueurs, soit plus aucun coup n'est possible.
	C'est la forme que nous allons utiliser durant tout ce projet, c'est pour cela que nous passerons rapidement sur
	les autres formes.

	\subsubsection{forme normale}
	La forme normale (ou stratégique) est définie par:
	\begin{itemize}
		\item L'ensemble des joueurs (de taille finie).
		\item L'ensemble des stratégies possibles pour chacun des joueurs (fini ou infini).
		\item Les préférences de chacun des joueurs, soit un sous-ensemble de stratégies parmi l'ensemble des
		\item combinaisons stratégiques possibles soit via une fonction d'utilité ou un fonction de gain.
	\end{itemize}

	\subsubsection{forme caractéristique}
	La forme caractéristique est utilisée, comme dit précédememnt, pour les jeux coopératifs. Elle est représentée
	sous forme d'un graphe G=(N,v) avec N l'ensemble des joueurs et v la fonction caractéristique. Cette dernière
	associe à chaque sous-ensemble de joueurs (noté S) qui forme une coalition, la valeur v(S), la gain obtenu
	par la coalition.

	\subsection{Recherche arborescente et intelligence artificielle}
	La recherche arborescente est une recherche qui se base sur la forme extensive afin d'utiliser l'arbre de
	décision\cite{cours_arbre_decision}.
	Cette recherche s'appuie aussi sur une fonction d'évaluation, qui associe à toutes situations du jeu une
	valeur indiquant si elle est favorable à une victoire. Bien évidemment plus cette valeur est grande plus
	la probabilité de victoire est grande. Cette fonction ne s'applique qu'aux feuilles de l'arbre de décision,
	ces feuille peuvent être la fin de la partie ou un état de la partie dans lequel nous évaluons le rapport de force
	entre les deux joueurs afin de déterminer qui est en meilleur posture pour gagner. Ainsi l'objectif de cette
	recherche est de choisir la suite de noeud à prendre (donc de coups à faire) afin d'arriver à la valeur
	d'évaluation maximale.

	\subsection{Méthode alpha-bêta}
	\subsubsection{Méthode min-max}
	Avant d'évoquer la méthode alpha-bêta, nous allons expliquer comment fonctionne la méthode min-max,
	la méthode alpha-bêta en est une amélioration. La méthode min-max est un algorithme utilisé dans la théorie des
	jeux, plus précisément dans les jeux à deux joueurs, à somme nulle et à information complète\cite{min_max}.

	L'objectif de cette dernière est de minimiser la perte dans le pire cas pour un joueur est de maximiser
	la perte pour l'adversaire, d'où son nom "min-max". Pour cela l'algorithme va simuler tous les coups
	possibles et leur donner une valeur, représentant combien ce coup est intéressant ou non.
	La simulation de tous les coups possibles passe par la construction de l'arbre de décision et
	le calcul de la valeur des feuilles est l'application d'une fonction d'évaluation. Nous obtenons donc
	un arbre avec des valeurs aux feuilles uniquement. Ensuite, nous allons devoir faire "remonter" la valeur
	du jeu des feuilles jusqu'à la racine et choisir parmi tous les coups celle qui a la valeur la moins basse. Pour
	calculer cette valeur nous allons passer par une approche récursive comme suit\cite{minimax_alog}:
	\begin{itemize}
		\item minmax(p) = f(p) si p est un feuille de l'arbre où f est la fonction d'évaluation.
		\item minmax(p) = max(minmax(O$_1$), \ldots, minmax(O$_n$)) si p est un noeud de l'ordinateur avec les
			fils O$_1$, \ldots, O$_n$.
		\item minmax(p) = min(minmax(O$_1$), \ldots, minmax(O$_n$)) si p est un noeud du joueur avec les fils
			O$_1$, \ldots, O$_n$.
	\end{itemize}

	Le souci de cette méthode est le nombre de cas à étudier. En effet comme la simulation
	produit un arbre que nous devons parcourir en entier afin de choisir le meilleur coup à jouer
	nous devons faire un parcours de cet arbre. Cependant comme cela passe soit par un algorithme récursif,
	qui doit simuler un nombre important de coups et de réponses ce qui fait augmenter la profondeur de l'arbre.
	En plus de faire augementer le nombre d'appels récursifs cela surcharge le tas d'appel. Soit, dans l'autre cas,
	nous pouvons passer par un algorithme itératif avec une file ou une pile. Or si le nombre de réponses possibles
	à un coup du joueur devient trop important cela nous oblige à utiliser beaucoup de mémoire. Enfin, même si nous
	prévoyons le fait que le programme utilise beaucoup de mémoire, nous ne voulons pas que le programme continue de
	calculer une suite de coups qui est moins avantageuse une suite qu'il à déjà calculé. Cela serai contre-productif.
	Ainsi c'est pour répondre à toutes ces problématiques que l'élagage alpha-bêta a été conçu.

	\subsubsection{Amélioration alpha-bêta}
	L'élagage alpha-bêta est comme son nom l'indique une méthode qui vise à "supprimer" les branches de l'arbre qui
	ne sont pas utiles, qui ne change rien au résultat final. Pour élager les branches, nous disposons de deux types de
	coupures, la coupure alpha et la coupure bêta, d'où le nom de la méthode\cite{elagage_alpha_beta}
	Cette méthode s'applique uniquement sur des arbres au moins de profondeur trois
	(la racine ayant un profondeur de un). Une fois que la méthode min-max à construit l'arbre
	et appliqué la fonction d'évaluation aux feuilles, nous allons ajouter les coupures lors de la
	"remonter" des valeurs à la racine.

	\subsubsection{Coupure Alpha}
	% TODO : Dessin à refaire
	\includegraphics[width=12cm]{images/elagageAlpha.JPG}

	Prenons l'exemple ci-dessus pour expliquer la coupure alpha, elle se place du point de vue du joueur.
	Une fois que la valeur du noeud B à été initialisé avec le minimum de ces fils (D, F) cette valeur va nous
	servir de borne pour faire nos coupures. Regardons les fils du frère de B, c'est à dire F, G (supposons que
	la valeur de H ne soit encore connue) afin d'initialiser la valeur de C. Nous devons prendre la valeur minimale
	entre F et G (ici F=4) mais nous constatons que cette valeur est plus petite que notre borne (B=5), ainsi explorer
	le noeud H ne sert a rien car peut importe la valeur qu'il aura (plus petite ou plus grande que F) il ne sera pas
	utiliser pour initialiser la valeur de A pusique A=max(B,C). Nous pouvons donc élaguer H (et tous les autres fils
	de C s'ils existent) comme sur l'image ci-dessous.

	% TODO : Dessin à refaire
	\includegraphics[width=12cm]{images/elagageAlphaSuite.JPG}

	\subsubsection{Coupure Bêta}
	La coupure bêta est similaire mais se base du point de vue de l'ordinateur, on élague donc les
	feuilles plus grande que la borne\cite{wiki_7_wonder}.

	\section{Travail effectué}
	\subsection{La fonction d'évaluation}
	Comme expliqué précédememnt la fonction d'évaluation nous permet d'associer un nombre à une situation 
	lors d'une partie. C'est-à-dire l'ensemble
	des éléments (cartes, merveilles, monnaies, position du jeton conflit, jetons militaires, jetons progrès)
	présents sur le plateau. Pour cela nous avons commencé par réduire le nombre d'éléments que nous prenons en
	compte lors d'une partie et nous avons travaillé uniquement avec les cartes. De plus nous avons considéré
	uniquement les cartes de l'age I. Après quelques brain storming nous avons conclu que la solution la
	plus simple était de donner un "poids" aux cartes en fonction de notre manière de jouer, puis de modifier ce poids
	en fonction de la situation.

	Pour définir ce poids, nous nous sommes tous reunis afin de nous concerter pour trouver la note la plus adequate
	pour chacune des cartes. Premièrement nous avons défini un intervalle de valeur arbitraire (entre 0 et 20)
	et nous avons aussi attribué un poids à une carte face cachée (la moyenne de l'intervalle donc 10).
	Pour les cartes avec une valeur	inférieure à 10, il est préférable de tenter de prendre la carte face cachée,
	avec une chance de 50 \% d'avoir
	une meilleure carte. De cette manière nous avons pu prendre en compte l'évolution des âges au cours d'une partie.
	Nous avons défini un objectif par âge, le premier doit permettre au joueur de collecter des ressources simples
	donc du bois, de la pierre et de l'argile. L'âge II doit permettre au joueur de collecter des ressources rares, donc
	verre et papyrus mais aussi attaquer et obtenir des symboles scientifiques et construire des merveilles. Durant
	le dernier âge le joueur doit viser les cartes donnant des points militaires.
	Cette solution nous permet aussi de prendre
	en compte le coût des cartes, par exemple les cartes "Chantier" et "Exploitation" donnent toutes les deux du bois
	mais "Exploitation" coûte une pièce, si nous avons le choix entre les deux il est préférable de prendre "Chantier"
	et d'utiliser la monnaie pour acheter des ressources que nous ne possédons pas.

	% TODO : détail à ajouter
	% TODO : mettre image du tableau

	Après plusieurs parties contre l'ordinateur, nous avons ajouté la possibilité de construire des merveilles.
	Nous avons utilisé la même méthode évoquée précédemment.

	% TODO : à finir

	\subsection{Implémentation}
	Pour l'implémentation nous avons décidé de commencer par réfléchir au découpage du jeu
	en différent modules ainsi que du design de l'interface.

	Nous avons conclus que l'arborescence des fichiers serait:
	\begin{itemize}
		\item Un dossier "interface\_graphique" contenant tous les fichiers utiles pour l'interface graphique, c'est-à
		-dire :
		\begin{itemize}
			\item Un dossier "ressources" avec toutes les images (cartes, merveilles, jetons, plateau, \ldots). Mais
			aussi les musiques.
			\item Un dossier "src" contenant les fichiers sources de l'interface, c'est-à-dire :
			\begin{itemize}
				\item Boutons.py
				\item Constantes.py
				\item Fenetre.py
				\item MonSprite.py
				\item SpriteCarte.py
				\item SpriteJetonMilitaire.py
				\item SpriteJetonProgre.py
				\item SpriteMerveille.py
			\end{itemize}
		\end{itemize}
		\item Un dossier "utils" contenant tous les fichiers utiles au jeu, c'est-à-dire :
		\begin{itemize}
			\item Carte.py
			\item CarteGuilde.py
			\item Merveille.py
			\item JetonMilitaire.py
			\item JetonProgres.py
			\item Joueur.py
			\item notation\_carte.cvs (un fichier stockant les "poids" des cartes)
			\item Outils.py
			\item Plateau.py
			\item Stratégie.py
		\end{itemize}
		\item Un dossier "test" contenant tous les fichiers tests, c'est-à-dire :
		\begin{itemize}
			\item TestCarteJeton.py
			\item TestJoueur.py
			\item TestMerveille.py
			\item TestPlateau.py
			\item TestStratégie.py
		\end{itemize}
	\end{itemize}

	Comme vous pouvez le constater les extensions de fichiers sont en ".py" cela signifie que se sont des fichiers
	Python. En effet lorsque nous avons discuté au sujet du langage à utiliser nous avions remarqué qu'ils nous fallait un
	langage orienté objet et que nous maîtrisons. Nous avions donc le choix entre le Java et le Python.
	Pour des raisons de simplicité nous avons choisi le Python. Une fois tous les préparatifs terminés nous avons
	commencé l'implémentation à proprement parlé. Nous avons suivi plusieurs phases, nous travaillons sur
	un module, une fois fini nous le testons à travers différentes batteries de tests, puis nous passons
	au module suivant. Une fois le jeu fini nous avons testé le jeu en 1vs1 via le terminal.

	Nous allons maintenant expliquer brièvement le contenue des différentes classes 
	constituant le projet et nous detaillerons le contenue de la classe "Stratégie" qui 
	est le coeur du projet.

	\subsubsection{Carte}
		Commençons par la classe carte.
		Une carte est composé d'un nom, d'un effet, d'un cout, d'un nom de carte de chainage (none si la carte n'en possède
		pas), une couleur, et elle appartient à un age.
		Concernant les effets et les couts, nous avons utilisé les propriétés du Python en définissant les effets
		comme des chaînes de caractère respectant un pattern précis. Elle commencent par le type d'effet
		comme "ressource", "monnaie", "attaquer", "symbole\_scientifique" puis les paramètres comme
		"ressource bois 1", "attaquer 3".
		Les classe "CarteGuilde" et "Merveille" sont des classes filles de la classe "Cartes". Les cartes guildes
		et les merveilles sont des cartes
		mais sans carte de chaînage sans couleur et sans âge (elles appatiennent toutes à l'âge III). Les merveilles
		possèdent un attribut pour savoir si elles sont construit ou non.

	\subsubsection{Jetons}
		Passons maintenant aux jetons. Il en existe deux types, le jeton militaire qui possède un nom, un nombre de pièces
		que perd l'adversaire lorsque le jeton est pris et un nombre de points de victoire que gagne le joueur.
		Le deuxième type est le jeton progrès qui possède un nom et des effets.

	\subsubsection{Joueur}
		Nous avons aussi une classe joueur. Le joueur est définit par un nom, et possède plusieurs listes
		d'objets visible sur le plateau, c'est-à-dire ces cartes, ces meveilles, et ces jetons progrès, ainsi 
		que de la monnaie. Mais aussi d'autre informations comme un dicionnaire de ressource, des points de victoires
		un dicionnaire de symbole scientifiques et le nombre de symbole scientifiques différent. Cette classe
		implémente certaines méthodes qui le jeu utilise pour avoir des informations comme par exemple
		les coûts manquants pour construire une carte, dans ce cas il faut regarder dans le dicionnaire de ressource
		du joueur, mais aussi ces monnaies ainsi que dans ces merveilles ou cartes la présence de carte donnant des
		ressources au choix ou d'autre effet impliquant la construction de carte ou de merveille. 
		Il existe aussi d'autres méthodes comme savoir si le joueur possède une carte pour faire du chaînage
		, s'il produit un type de ressources en particulier, ect\ldots.

	\subsubsection{Plateau}
		Toutes les différentes fonctionnalitées du jeu sont dans le fichier Plateau.py. Cette classe
		représente le jeu dans sa globalité c'est-à-dire le plateau avec les jetons progrès choisit,
		les cartes présentent en dessous du plateau, les deux joueurs, la fausse et la banque.
		Elle s'occupe aussi des effets des cartes ainsi que des effets des merveilles.
		C'est aussi la première classe que nous avons developpé durant le projet, elle a donc connue
		plusieurs évolutions.
		La classe Plateau contient toutes les méthodes faisant fonctionner le jeu. Elle 
		implémente une méthode pour préparer le plateau, c'est-à-dire
		tirer aléatoirement les jetons, les cartes, ainsi que les merveilles (si le jeu
		est en mode choix automatique). Elle contient aussi des méthodes pour piocher, 
		pour enlever une carte, pour obtenir la liste des cartes prenables, pour intéragir
		avec la banque, ect \ldots.

		Après tout cela, l'équipe se sépare en deux groupes, l'un crée l'interface pendant que l'autre fait
		l'algorithme minimax, d'alpha beta ainsi que la fonction d'évaluation.

	\subsubsection{Stratégie}
		Concernant la classe "Stratégie", elle contient la fonction d'évaluation, la fonction
		minimax ainsi que son amélioration la fonction alpha-beta.
		Elle commence d'abord par lire le fichier contenant le poids des cartes, des merveilles, et des jetons progrès.
		
		\begin{lstlisting}
			notation_carte = {}
			with open("src/utils/notation_cartes.csv", mode='r') as file:
				fichier_cvs = csv.reader(file)
				for lignes in fichier_cvs:
					notation_carte[lignes[0]] = int(lignes[1])
		\end{lstlisting}
		
		Nous avons rempli un fichier Google Sheet avec le nom de la carte puis son poids. Puis nous l'avons importé
		sous un format rapide et simple à lire, le format csv. Le Python nous a permit d'éxecuter ceci en seulement
		quelques lignes de code.

		Nous avons ensuite la fonction d'évaluation, elle prend en entrée
		la partie à un instant donné et va renvoyé une valeur, la différence
		entre l'évalution du joueur 2, donc de l'ordinateur, et du joueur 1.

		\begin{lstlisting}
			def fonction_evaluation(partie):
				evaluation_j2 = 0
				for carte in partie.joueur2.cartes:
					if carte.est_face_cachee:
						evaluation_j2 += 10
					else:
						evaluation_j2 += notation_carte[carte.nom]
				
				for merveille in partie.joueur2.liste_merveilles_construite():
					evaluation_j2 += notation_carte[merveille.nom]
				
				if partie.joueur1.monnaie == 0:
					evaluation_j2 += 10
								
				if partie.position_jeton_conflit == 0:
					evaluation_j2 += 10
				
				evaluation_j1 = 0
				for carte in partie.joueur1.cartes:
					if carte.est_face_cachee:
						evaluation_j1 += 10
					else:
						evaluation_j1 += notation_carte[carte.nom]
				
				for merveille in partie.joueur1.liste_merveilles_construite():
					evaluation_j1 += notation_carte[merveille.nom]
				
				if partie.joueur2.monnaie == 0:
					evaluation_j1 += 10
				
				if partie.position_jeton_conflit == 18:
					evaluation_j1 += 10
				return evaluation_j2 - evaluation_j1
		\end{lstlisting}

		Comme nous l'avons expliqué plutôt la fonction va partir courir les cartes du joueur deuxième
		et va faire la somme de leur poids. Puis elle va faire
		la somme du poids des merveilles construites. Elle va de même pour le joueur un.
		Nous avons ajouté deux trois modifications, notamment si la position du jeton militaire
		est à 0, donc dans la ville du joueur 1, la situation est très favorable pour l'ordinateur
		est inversement.

		Enfin nous avons la fonction minimax et alpha-beta, nous sommes parties d'un tutorielle
		sur Youtube \cite{implementation_minimax_alpha_beta} que nous avons adpté avec notre jeu.
		La fonction minimax est une fonction récursive qui prend en paramètre la partie, 
		la profondeur a calculé, un booléen indiquant si c'est à l'ordinateur de joueur et le nombre
		de noeuds. Elle renvoie l'évaluation du meilleur coup, la carte à prendre, 
		et le nombre de noeuds évalué.

		Nous allons commencer par expliquer l'implémentation avec la version 
		qui ne prend pas en compte les merveilles et les jetons progrès.
		\begin{lstlisting}
			def minimax(partie, profondeur, coup_bot, nbr_noeuds):
				if profondeur == 0 or partie_fini(partie):
					return fonction_evaluation(partie), None, nbr_noeuds+1
				
				carte_a_prendre = None
				
				if coup_bot:
					partie.joueur_qui_joue = partie.joueur2
					max_eval = -math.inf
					
					for carte in partie.liste_cartes_prenables():
						
						copie_partie: Plateau = partie.constructeur_par_copie()
						ret = copie_partie.piocher(carte)
						
						if ret == -1:
							copie_partie.defausser(carte)
						else:
							copie_partie.joueur_qui_joue.cartes.append(carte)
							copie_partie.enlever_carte(carte)
						
						evaluation, _, nbr_noeuds = minimax(copie_partie, profondeur - 1, False, nbr_noeuds)
						
						if evaluation > max_eval:
							max_eval = evaluation
							carte_a_prendre = carte
							
					return max_eval, carte_a_prendre, nbr_noeuds+1
				
				else:
					partie.joueur_qui_joue = partie.joueur1
					min_eval = math.inf
					
					for carte in partie.liste_cartes_prenables():
						
						copie_partie: Plateau = partie.constructeur_par_copie()
						copie_partie.piocher(carte)
						copie_partie.joueur_qui_joue.cartes.append(carte)
						copie_partie.enlever_carte(carte)
					
						evaluation, _, nbr_noeuds = minimax(copie_partie, profondeur - 1, True, nbr_noeuds)
						
						if evaluation < min_eval:
							min_eval = evaluation
							carte_a_prendre = carte
					
					return min_eval, carte_a_prendre, nbr_noeuds+1
		\end{lstlisting}

		Cette fonction permet de simuler le coup de l'ordinateur et le coup du joueur.
		Pour cela elle parcourt toutes la carte prenable de la partie passée en paramètre,
		pour chacune la fonction créer une copie de la partie. Dans cette copie
		de la partie le joueur qui joue pioche la carte, ce qui applique les effets 
		de cette dernière. Puis nous appelons recursivement la fonction avec la partie
		copie en paramètre. Dans cettee copie nous regardons a nouveau les cartes prenables
		et ainsi de suite nous construisons l'arbre de décision.

		Passons maintenant à la version alpha-beta de la fonction.
		\begin{lstlisting}
			
		\end{lstlisting}

	\subsection{Interface}
	\subsection{Résultat obtenu}

	\section{Conclusion}
	\subsection{Objectif(s) atteint/ non atteint}
	\subsection{Suggestion d'amélioration(s)}
	\subsection{Difficultée(s) rencontrée(s)}
	Nous sommes rapidement tombés sur une première difficultée lorsque nous rédigions le schéma de classe du jeu.
	Par exemple, les cartes comportent beaucoup d'effets pouvant avoir un impact sur le joueur,
	comme un gain de ressource, mais aussi sur le jeu, comme le fait du pouvoir rejouer. L'application d'un effet
	doit être effectué par le jeu et non dans le module "Carte". Ensuite nous voulions créer des classes (plus
	précisément des interfaces) pour les effets mais cela rendait compliqué la création de carte avec les effets.
	% TODO : idée : constructeur par copie

	\section{Annexe}
	\subsection{Code}

	\printbibliography

\end{document}
